%! Author = Anastasia
%! Date = 10.04.2023

% Preamble
\documentclass[a4paper, 12pt, oneside]{scrartcl}
% Packages
\usepackage[T2A]{fontenc}
\usepackage[english, russian]{babel}
\usepackage{amsmath}
\usepackage{graphicx}
\graphicspath{{./Images/}}
\usepackage[utf8]{inputenc}
\DeclareGraphicsExtensions{.png,.jpg}
% Document
\title{Хеширование строк}
\author{}
\date{}
\begin{document}
    \maketitle
    \section{Хеш-функции}\label{sec:section1}
   Хеш-функция предназначена для свертки входного массива любого размера в битовую строку, для MD5 длина выходной строки равна 128 битам. Для чего это нужно? К примеру у вас есть два массива, а вам необходимо быстро сравнить их на равенство, то хеш-функция может сделать это за вас, если у двух массивов хеши разные, то массивы гарантировано разные, а в случае равенства хешей — массивы скорее всего равны.
   
   Хеш-функции используются в криптографических алгоритмах, электронных подписях, кодах аутентификации сообщений, обнаружении манипуляций, сканировании отпечатков пальцев, контрольных суммах (проверка целостности сообщений), хеш-таблицах, хранении паролей и многом другом. К примеру, скачивая файл из интернета, вы часто видите рядом с ним строку вида b10a8db164e0754105b7a99be72e3fe5 — это и есть хеш, прогнав этот файл через алгоритм MD5 вы получите такую строку, и, если хэши равны, можно с большой вероятностью утверждать что этот файл действительно подлинный (конечно с некоторыми оговорками, о которых расскажу далее).~\cite{managementsystem}

    \section{MD5}\label{sec:section2}
    Default2.
    \section{SHA-1}\label{sec:section3}
    Default3.
    \bibliography{main}
    \bibliographystyle{plain}

\end{document}
