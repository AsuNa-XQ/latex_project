%! Author = Anastasia
%! Date = 10.04.2023

% Preamble
\documentclass[a4paper, 12pt]{article}
% Packages
\usepackage[T2A]{fontenc}
\usepackage[english, russian]{babel}
\usepackage{amsmath}
\usepackage[utf8]{inputenc}
% Document
\begin{document}

    \section{Section1}\label{sec:introduction}
    Привет.~\cite{managementsystem}

    Я в разделе~\ref{sec:introduction}
	Под техническим риском понимается значение риска информационной безопасности, состоящего из вероятностей реализации угроз и использования уязвимостей каждого компонента
	информационной инфраструктуры с учетом уровня их конфиденциальности, целостности и доступности.
    Hellow.
    \section{Section2}\label{sec:introduction2}
    В классическом представлении риск – это вероятность реализации угрозы информационной безопасности.
    Оценка риска производится на основе анализа ценности ИТ-актива для бизнеса, уязвимостей, угроз и вероятности их реализации.
    \bibliography{main}
    \bibliographystyle{plain}

\end{document}
